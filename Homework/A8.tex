\documentclass{article}
\usepackage{amssymb}
\usepackage{amsmath}
\usepackage{centernot}

\begin{document}

\title{M328K\: Homework 8}
\author{Joshua Dong}
\date{\today}
\maketitle


%4.4.6, 4.4.10, 4.6.2(b), 5.1.2

\subsection{4.4.6}
Find all solutions of $x^8-x^4+1001 \equiv 0 \pmod{539}$.
\\
\\$x^8-x^4+462 \equiv 0 \pmod{7^211}$.
\\$\therefore x^8-x^4+462 \equiv 0 \pmod{7^2}$ and
$x^8-x^4+462 \equiv 0 \pmod{11}$
\\We first find solutions to the first congruence.
\\$x^8-x^4+21 \equiv 0 \pmod{7^2}$
\\First, we find solutions to $x^8-x^4+21 \equiv 0 \pmod{7}$
\\$x^8-x^4 \equiv 0 \pmod{7}$
\\$x^4(x^4-1) \equiv 0 \pmod{7}$
\\The soultions modulo 7 are 0, 1, and 6.
\\To see if solutions lift, we can use Hansel's Lemma.
\\$f(x) = x^8-x^4+21$.
\\$f'(x) = 8x^7-4x^3 \equiv x^7-4x^3 \pmod7$.
\\$f'(0) \equiv 0 \pmod7$.
\\$f(0) \not\equiv 0 \pmod{49}$.
\\$\therefore x \equiv 0 \pmod 7$ does not lift modulo 49.
\\$f'(1) \equiv 1-4 \equiv -3 \pmod7$.
\\$f(1) \equiv 1-1+21 \equiv 21 \pmod49$.
\\$\therefore x \equiv 1 \pmod 7$ lifts to 8 modulo 49.
\\$f'(-1) \equiv (-1)-(-4) \equiv 3 \pmod7$.
\\$f(-1) \equiv 1-1+21 \equiv 21 \pmod7$.
\\$\therefore x \equiv 1 \pmod 7$ lifts to 41 modulo 49.
\\
\\We now find solutions to the second congruence.
\\$x^8-x^4+0 \equiv 0 \pmod{11}$
\\$x^4(x^4-1) \equiv 0 \pmod{11}$
\\$\therefore x \equiv 0, 1, 10 \pmod{11}$
\\
\\The intersection is found by solving the linear system of congruences.
\\Therefore, the solution set is $\{x \mid x \equiv 8 \pmod{49}$ or
$x \equiv 41 \pmod{49}\}$

\newpage
\subsection{4.4.10}
Find all solutions of $x^5+x-6 \equiv 0 \pmod{2^43^2}$.
\\
\\We first find soultions to $x^5+x-6 \equiv 0 \pmod{2^4}$ and $x^5+x-6 \equiv 0 \pmod{3^2}$
\\$x^5+x-6 \equiv 0 \pmod{2^4}$
\\$x^5+x-6 \equiv 0 \pmod{3^2}$.
\\Solving the first congruence:
\\$x^5+x+2 \equiv 0 \pmod{4}$
\\By observation, 0 is not a solution, but 1, 2, and 3 are modulo 4.
\\For some $t \in \mathbb{Z}$, $(4t+1)^5+(4t+1)-6 \equiv 1+20t+4t-5
\equiv 8t-4 \equiv 0 \pmod{16}$. But this is never true, thus 1 does not lift.
\\By trying all the rest of the possible solutions, we find that
x is 3, 6, or 11 modulo 16.
\\
\\Solving the second congruence:
\\By observation, 0, 1, -1, 2, and -2 are not solutions. 
\\$3^5+3-6 = 81(3)+3-6 = 240$. 3 is not a solution.
\\$(-3)^5-3-6 = -(240+12) = 252$. -3 is a solution. (2+5+2 = 9)
\\$(4)^5+4-6 = 256(4)-2 = 800+200+22$. 4 is not a solution.
\\$(5)^5+5-6 = 625(5)-1 = 3000+124$. 5 is not a solution.
\\
\\Solving the system:
\\$x \equiv 6 \pmod9
\\x \equiv 3, 6, 11 \pmod{16}$.
\\x is in the form $6+144t$, $51+144t$, or $123+144t$ for any $t \in \mathbb{Z}$.

\subsection{5.1.2}
Determine the highest power of 2 that divides 1423408.
\\
$1423408 = 711704(2) = 355852(2^2) = 177926(2^3) = 88963(2^4)$.
The highest power of 2 is 16.

\end{document}
