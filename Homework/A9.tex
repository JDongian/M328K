\documentclass{article}
\usepackage{amssymb}
\usepackage{amsmath}
\usepackage{centernot}
\setlength{\parindent}{0in}

\begin{document}

\title{M328K\: Homework 8}
\author{Joshua Dong}
\date{\today}
\maketitle

\subsection{6.1.16}
Show that if n is a composite integer with $n \neq 4$,
then $(n-1)! \equiv 0 \pmod{n}$
\\
\\If n is composite, then there two possibilities:
\\Case 1:
\\$\exists a,b \in \textbf{P}$ where $ab = n$ 
and $2 \le a < b \le n-2$.
\\$(ab-1)! =
1\cdot2\cdot3\cdot\ldots\cdot a \cdot\ldots b \cdot\ldots\cdot (ab-1)
\rightarrow ab \mid (ab-1)!$
\\
\\Case 2:
\\$\nexists a,b \in \textbf{P}$ where $ab = n$ and $2 \le a < b \le n-2
\rightarrow \exists p \in \textbf{P}$ where $p^2 = n$.
\\Since $n>p$, $(n-1)! =
1\cdot2\cdot3\cdot\ldots\cdot p \cdot\ldots\cdot (n-1)$.
\\$\therefore p \mid (n-1)!$
\\
\\Since $n \neq 4$ and all positive squares are greater than 4, $2p < n$.
\\$\therefore (n-1)! =
1\cdot2\cdot3\cdot\ldots\cdot 2p \cdot\ldots\cdot (n-1)$.
\\$\therefore 2p \mid (n-1)! \rightarrow
2p^2 \mid (n-1)! \rightarrow
2n \mid (n-1)! \rightarrow
n \mid (n-1)!$
\\
\\$\therefore (n-1)! \equiv 0 \pmod{n}$ for all composites n, $n\neq4$.

\subsection{6.1.22}
Show that $30 \mid (n^9 - n)$\;\;$ \forall n \in \mathbb{Z}^+$.
\\
\\$2,3,5 \mid n(n-1)(n+1)(n^2+1) \rightarrow
\\30 \mid n(n-1)(n+1)(n^2+1) \rightarrow
\\30 \mid n(n-1)(n+1)(n^2+1)(n^4+1) \rightarrow
\\30 \mid (n^9 -n)$.
\\We now are left to prove that 2, 3, and 5 divide $n(n-1)(n+1)(n^2+1)$.
\newpage
 $n(n-1)$ forms a sequence of two consecutive integers.
\\Therefore, one of them must divide 2
(this could be trivially shown with an enumeration of possibilities
in the form $2q+r$).
\\$\therefore 2 \mid n(n-1)(n+1)(n^2+1)$\;\;$ \forall n \in \mathbb{Z}^+$.
\\
\\$n(n-1)(n+1)$ forms a sequence of three consecutive integers.
\\We can use the same argument we used to prove divisibility by two.
\\$\therefore 3 \mid n(n-1)(n+1)(n^2+1)$\;\;$ \forall n \in \mathbb{Z}^+$.
\\
\\Suppose 5 does not divide $k(k-1)(k+1)$.
\\k must then be in the form $5t+2$ or $5t+3$ for some $t \in \mathbb{Z}^+$,
as a form of $5t+0$, $5t+1$, or $5t+4$
would result in the product having a term divisible by 5.
\\If k is in the form $5t+2$ or $5t+3$, then $(k^2+1)$ is in the form
$25t+4+1$ or $25t+9+1$, both which are divisible by 5.
\\$\therefore 5 \mid n(n-1)(n+1)(n^2+1)$\;\;$ \forall n \in \mathbb{Z}^+$.
\\
\\$\therefore 30 \mid (n^9 - n)$\;\;$ \forall n \in \mathbb{Z}^+$.
\\(However, it is worth noting that if $n<2$, then the product is 0)

\subsection{6.3.4}
Show that if $a, m \in \mathbb{Z}^+$, $(a,m)=(a-1,m)=1$,
then $1+a+a^2+...+a^{\varphi(m)-1}\equiv0\pmod{m}$.
\\
\\Let $x \in \mathbb{Z}$ where $x = 1+a+a^2+...+a^{\varphi(m)-1}$
\\$ax = a+a^2+...+a^{\varphi(m-1)}+a^{\varphi(m)}$
\\$ax+1 = 1+a+a^2+...+a^{\varphi(m-1)}+a^{\varphi(m)}$
\\$ax+1 = x+a^{\varphi(m)}$
\\$x = \frac{a^{\varphi(m)}-1}{a-1}$
\\$1+a+a^2+...+a^{\varphi(m)-1} \equiv \frac{a^{\varphi(m)}-1}{a-1} \pmod{m}$
\\$\frac{a^{\varphi(m)}-1}{a-1} \equiv \frac{1-1}{a-1} \equiv 0 \pmod{m}$,
by Euler's totient theorem, given $a > 1$.
\\$\therefore 1+a+a^2+...+a^{\varphi(m)-1}\equiv0\pmod{m}$
\;\;$\forall a, m \in \mathbb{Z}^+$, $(a,m)=(a-1,m)=1$.

\subsection{6.3.10}
Show that $a^{\varphi(b)}+b^{\varphi(a)}\equiv1\pmod{ab}$ given $a \perp b$.
\\
\\$a^{\varphi(b)}+b^{\varphi(a)}-1\equiv
b^{\varphi(a)}-1 \equiv
0 \pmod{a}$ by Euler's totient theorem ($a \perp b$).
\\Without loss of generality,
$a^{\varphi(b)}+b^{\varphi(a)}-1\equiv0\pmod{b}$.
\\$\therefore a^{\varphi(b)}+b^{\varphi(a)}-1\equiv0\pmod{ab}$.
\\$\therefore a^{\varphi(b)}+b^{\varphi(a)}\equiv1\pmod{ab}$
for all coprime integers a and b.

\end{document}
