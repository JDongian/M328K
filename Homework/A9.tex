\documentclass{article}
\usepackage{amssymb}
\usepackage{amsmath}
\usepackage{centernot}

\begin{document}

\title{M328K\: Homework 8}
\author{Joshua Dong}
\date{\today}
\maketitle

\subsection{6.1.16}
Show that if n is a composite integer with $n \neq 4$,
then $(n-1)! \equiv 0 \pmod{n}$
\\
\\If n is composite, then there two possibilities:
\\Case 1:
\\There exist different prime integers a and b where $ab = n$ 
and $2 \le a < b \le n-2$.
\\Both a and b are in the product $(ab-1)!$.
\\Therefore $(ab-1)!$ is divisible by ab.
\\
\\Case 2:
\\If n cannot be expressed as a product of two different primes,
then n is a square where $n = p^2$.
\\p appears in the factorial product of $(n-1)!$, therefore $p \mid (n-1)!$.
\\We can say $2p \mid (n-1)!$ if $2p < n$,
which is true for all squares where $n \neq 4$.
\\If $2p \mid (n-1)!$ and $p \mid (n-1)!$, then $2p^2 \mid (n-1)!$.
\\Therefore $2n \mid (n-1)!$.
\\Therefore $n \mid (n-1)!$.
\\
\\Therefore, $(n-1)! \equiv 0 \pmod{n}$ for all composites n, $n\neq4$.
\newpage

\subsection{6.1.22}
Show that $30 \mid (n^9 - n)$\;\;$ \forall n \in \mathbb{Z}^+$.
\\$(n^9 - n) = n(n-1)(n+1)(n^2+1)(n^4+1)$
\\If we can show that $30 \mid n(n-1)(n+1)(n^2+1)$,
\\then $30 \mid n(n-1)(n+1)(n^2+1)(n^4+1)$.
\\If we can show that 2, 3, and 5 divide $n(n-1)(n+1)(n^2+1)$,
\\then 30 divides $n(n-1)(n+1)(n^2+1)$.
\\$n(n-1)$ is always even (the product of an even and odd is odd).
\\Therefore 2 divides $n(n-1)(n+1)(n^2+1)$\;\;$ \forall n \in \mathbb{Z}^+$.
\\
\\$n(n-1)(n+1)$ forms a sequence of three consecutive integers.
\\Therefore, one of them must divide 3
(this could be shown with an enumeration of possibilities)
\\Therefore 3 divides $n(n-1)(n+1)(n^2+1)$\;\;$ \forall n \in \mathbb{Z}^+$.
\\
\\Suppose 5 does not divide $k(k-1)(k+1)$.
\\k must then be in the form $5t+2$ or $5t+3$ for some $t \in \mathbb{Z}^+$,
as a form of $5t+0$, $5t+1$, or $5t+4$
would result in the product having a term divisible by 5.
\\If k is in the form $5t+2$ or $5t+3$, then $(k^2+1)$ is in the form
$25t+4+1$ or $25t+9+1$, both which are divisible by 5.
\\Therefore 5 divides $n(n-1)(n+1)(n^2+1)$\;\;$ \forall n \in \mathbb{Z}^+$.
\\
\\Therefore 30 divides $(n^9 - n)$\;\;$ \forall n \in \mathbb{Z}^+$.
\\However, it is worth noting that if $n<2$, then the product is 0.

\subsection{6.3.4}
Show that if $a, m \in \mathbb{Z}^+$, $(a,m)=(a-1,m)=1$,
then $1+a+a^2+...+a^{\phi(m)-1}\equiv0\pmod{m}$.
\\
\\We split the proof into two cases.
\\Case 1: m is prime.
\\$\phi(m) = m-1$.
\\$\therefore$ there are m-1 terms in the sequence.
\\If we can show that the series forms a complete set of residues modulo m
(the 0 term is omissible),
then that is sufficient to prove the assertion,
as the sum of all residues is 0 modulo m.
\\We can rewrite each term with respect to the reverse index $i$
from $a^{\phi(m)-i}$ to $\bar{i}$.
\\$\bar{i}$ exists because m is prime, and is unique for each i.
\\Therefore, for prime moduli m, the sum is equivalent to 0 modulo m.
\\Case 2: m is composite.

\subsection{6.3.10}
--
\\

\end{document}
