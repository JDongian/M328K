\documentclass{article}
\usepackage{amssymb}
\usepackage{amsmath}
\usepackage{centernot}
\setlength{\parindent}{0in}


\usepackage{color}
\usepackage{listings}
\lstset{ %
    language=Python,                % choose the language of the code
    basicstyle=\footnotesize,       % the size of the fonts that are used for the code
    numbers=left,                   % where to put the line-numbers
    numberstyle=\footnotesize,      % the size of the fonts that are used for the line-numbers
    stepnumber=1,                   % the step between two line-numbers. If it is 1 each line will be numbered
    numbersep=5pt,                  % how far the line-numbers are from the code
    backgroundcolor=\color{white},  % choose the background color. You must add \usepackage{color}
    showspaces=false,               % show spaces adding particular underscores
    showstringspaces=false,         % underline spaces within strings
    showtabs=false,                 % show tabs within strings adding particular underscores
    frame=single,           % adds a frame around the code
    tabsize=2,          % sets default tabsize to 2 spaces
    captionpos=b,           % sets the caption-position to bottom
    breaklines=true,        % sets automatic line breaking
    breakatwhitespace=false,    % sets if automatic breaks should only happen at whitespace
    escapeinside={\%*}{*)}          % if you want to add a comment within your code
}

\begin{document}
\title{M328K\: Homework 13}
\author{Joshua Dong}
\date{\today}
\maketitle

\subsection{9.3.8}
Find a primitive root for 6, 18, 26, and 388.
\\
\begin{lstlisting}
x = int(raw_input("Find primitive root for: "))

def primes(n):
    primfac = set([])
    d = 2
    while d*d <= n:
        while (n % d) == 0:
            primfac.add(d)
            n /= d
        d += 1
    if n > 1:
        primfac.add(n)
    return primfac

def is_coprime(a, b):
    for p in primes(min(a,b)):
        if max(a,b)%p == 0:
            return False
    return True

def is_complete(g, x):
    complete_mult_group = set(range(1,x))
    for e in list(complete_mult_group):
        if not is_coprime(e, x):
            complete_mult_group.remove(e)
    return g == complete_mult_group

roots = []
for i in xrange(2,x):
    group = set([])
    for j in xrange(1,x):
        result = (i**j)%x
        if not result in group:
            group.add(result)
        else:
            break
    if is_complete(group, x):
        roots.append(i)
print roots
\end{lstlisting}
We use this program to find all the primitive roots for any integer.
\\6 has only: \[5\]
\\18 has: \[5, 11\]
\\26 has: \[7, 11, 15, 19\]
\\388 has none. 

\subsection{9.3.12}
Show that there are the same number of primitive roots modulo $2p^t$ as there are modulo $p^t$, where $p$ is an odd prime and $t \in \mathbb{Z}^+$.
\\
\\Using Theorem 9.14 from the book, we know that for all primitive roots modulo $p^t$ there exists a corresponding primitive root modulo $2p^t$.
\\We want to show that for all primitive roots modulo $2p^t$ there exists a corresponding primitive root modulo $p^t$.
\\Let $r$ be a primitive root modulo $2p^t$ where $p$ is an odd prime and $t \in \mathbb{Z}^+$.
\\Then $r^{\varphi(2p^t)} \equiv 1 \pmod{2p^t}$ where $\varphi(2p^t)$ is the lowest exponent such that this is true.
\\$\varphi(2p^t) = \varphi(2)\varphi(p^t) = \varphi(p^t)$.
\\$\therefore r^{\varphi(p^t)} \equiv 1 \pmod{2p^t}$.
\\$\therefore r^{\varphi(p^t)} \equiv 1 \pmod{p^t}$.
\\We want to show that no smaller power of $r$ is congruent to 1 modulo $p^t$.
\\If $r$ is greater than $p^t$, $r$ corresponds to the primitive root $r-p^t$ modulo $p^t$.
\\If $r$ is less than $p^t$, $r$ is a primitive root modulo $p^t$.
\\Therefore, for all primitive roots modulo $2p^t$ there exists a primitive root modulo $p^t$.
\\Therefore there exists a one-to-one correspondence from primitive roots of $2p^t$ to primitive roots of $p^t$.
\\Therefore there are the same number of primitive roots modulo $2p^t$ as there are modulo $p^t$, where $p$ is an odd prime and $t \in \mathbb{Z}^+$.

\end{document}
