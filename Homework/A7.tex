\documentclass{article}
\usepackage{amssymb}
\usepackage{amsmath}
\usepackage{centernot}

\begin{document}

\title{M328K\: Homework 7}
\author{Joshua Dong}
\date{\today}
\maketitle


%10, 20(b,e), 32, 36

% 4.3, problem 10
\subsection{4.3.10}
Find $x\in\mathbb{Z}$ such that\
\\$x\equiv9\pmod{10}$,
\\$x\equiv9\pmod{11}$, and
\\$x\equiv0\pmod{13}$.
\\
\\Using the first congruence,
\\Let $q \in \mathbb{Z}$ where $x=10q+9$.
\\Substituting into the second congruence,
\\$10q + 9 \equiv 9 \pmod{11}$.
\\It follows that $10q \equiv 0 \pmod{11}$,
\\$q \equiv 0 \pmod{11}$.
\\Let $r \in \mathbb{Z}$ where $q=11r+0$.
\\Now we have $x=10(11r)+9=110r+9$.
\\Substituting into the third congruence,
\\$110r + 9 \equiv 0 \pmod{13}$.
\\It follows that $110r \equiv 4 \pmod{13}$,
\\$6r \equiv 4 \pmod{13}$,
\\$r \equiv (\bar{6})(4) \equiv 44 \equiv 5 \pmod{13}$.
\\Let $s \in \mathbb{Z}$ where $r=13s+5$.
\\Now we have $x=110(13s+5)+9=1430s+559$.
\\559 is a vaild value for $x$.

\newpage
\subsection{4.3.20.b}
Find $x\in\mathbb{Z}$ such that\
\\$x\equiv2\pmod{14}$,
\\$x\equiv16\pmod{21}$, and
\\$x\equiv10\pmod{30}$.
\\
\\We derive an equivalent, minimal set of congruences
using the Chinese remainder theorem.
\\$x\equiv2\pmod{2}$
\\$x\equiv2\pmod{7}$
\\$x\equiv16\pmod{3}$
\\$x\equiv16\pmod{7}$
\\$x\equiv10\pmod{2}$
\\$x\equiv10\pmod{3}$
\\$x\equiv10\pmod{5}$
\\These reduce to:
\\$x\equiv0\pmod{2}$,
\\$x\equiv1\pmod{3}$,
\\$x\equiv0\pmod{5}$,
\\$x\equiv2\pmod{7}$.
\\
\\Using the first congruence,
\\Let $q \in \mathbb{Z}$ where $x=2q$.
\\Substituting into the second congruence,
\\$2q \equiv 1 \pmod{3}$.
\\It follows that $q \equiv 2 \pmod{3}$.
\\Let $r \in \mathbb{Z}$ where $q=3r+2$.
\\Now we have $x=2(3r+2)=6r+4$.
\\Substituting into the third congruence,
\\$6r + 4 \equiv 0 \pmod{5}$.
\\It follows that $r \equiv 1 \pmod{5}$.
\\Let $s \in \mathbb{Z}$ where $r=5s+1$.
\\Now we have $x=6(5s+1)+4=30s+10$.
\\Substituting into the fourth congruence,
\\$30s + 10 \equiv 2 \pmod{7}$.
\\It follows that $s \equiv (\bar{2})(6) \equiv 3 \pmod{7}$.
\\Let $t \in \mathbb{Z}$ where $s=7t+3$.
\\Now we have $x=30(7t+3)+10=210t+100$.
\\Thus any $x \in S$ where
$S = \{x \mid 210t+100 $\;\;$\forall t \in \mathbb{Z}\}$
will be a solution to the congruence.

\newpage
\subsection{4.3.20.e}
Find $x\in\mathbb{Z}$ such that\
\\$x\equiv7\pmod{9}$,
\\$x\equiv2\pmod{10}$,
\\$x\equiv3\pmod{12}$, and
\\$x\equiv6\pmod{15}$.
\\
\\We derive an equivalent, minimal set of congruences
using the Chinese remainder theorem.
\\$x\equiv7\pmod{9}$
\\$x\equiv2\pmod{2}$
\\$x\equiv2\pmod{5}$
\\$x\equiv3\pmod{3}$
\\$x\equiv3\pmod{4}$
\\$x\equiv6\pmod{3}$
\\$x\equiv6\pmod{5}$
\\These reduce to:
\\$x\equiv0\pmod{2}$,
\\$x\equiv0\pmod{3}$,
\\$x\equiv3\pmod{4}$,
\\$x\equiv1\pmod{5}$,
\\$x\equiv2\pmod{5}$,
\\$x\equiv7\pmod{9}$.
\\
\\There are no solutions to this contradictory set of congruences.
(The intersection of the system is the null set).

\newpage
\subsection{4.3.32}
Show that the system 
\\$x\equiv1\pmod{2}$,
\\$x\equiv0\pmod{4}$,
\\$x\equiv0\pmod{3}$,
\\$x\equiv2\pmod{12}$,
\\$x\equiv2\pmod{8}$,
\\$x\equiv22\pmod{24}$
\\is a covering set of congruences.
\\
\\We derive an equivalent set of congruences
using the Chinese remainder theorem.
\\$x\equiv1\pmod{2}$,
\\$x\equiv0\pmod{4}$,
\\$x\equiv0\pmod{3}$,
\\$x\equiv2\pmod{3}$,
\\$x\equiv2\pmod{4}$,
\\$x\equiv2\pmod{8}$,
\\$x\equiv22\pmod{3}$,
\\$x\equiv22\pmod{8}$.
\\These reduce to:
\\$x\equiv1\pmod{2}$,
\\$x\equiv0\pmod{3}$,
\\$x\equiv1\pmod{3}$,
\\$x\equiv2\pmod{3}$,
\\$x\equiv0\pmod{4}$,
\\$x\equiv2\pmod{4}$,
\\$x\equiv2\pmod{8}$,
\\$x\equiv6\pmod{8}$.
\\
\\The union of:
\\$x\equiv0\pmod{3}$,
\\$x\equiv1\pmod{3}$,
\\$x\equiv2\pmod{3}$
\\is $U$, the universal set of all integers.
\\Therefore the system is a covering set of congruences.
\\

\newpage
\subsection{4.3.36}
Find all solutions of the congruence $x^2+6x-31\equiv0\pmod{2^33^2}$.
\\
\\$x^2+6x-31\equiv0\pmod{2^3}$.
\\$x^2+6x+9\equiv0\pmod{8}$
\\$(x+3)^2\equiv0\pmod{8}$.
\\$\therefore x\equiv1\pmod{8}$ or
\\$x\equiv5\pmod{8}$.
\\These are equivalent to the congruence $x\equiv1\pmod{4}$.
\\
\\$x^2+6x-31\equiv0\pmod{3^2}$.
\\$x^2-3x-4\equiv0\pmod{9}$
\\$(x-4)(x+1)\equiv0\pmod{9}$.
\\$\therefore x\equiv4\pmod{9}$ or
\\$x\equiv8\pmod{9}$.
\\
\\We want the intersection of the modulo 8 and modulo 9 solutions:
\\$x\equiv1\pmod{4}$ and
\\$x\equiv4\pmod{9}$ or $x\equiv8\pmod{9}$.
\\This is equivalent to:
\\$x\equiv1\pmod{4}$ and $x\equiv4\pmod{9}$ or
\\$x\equiv1\pmod{4}$ and $x\equiv8\pmod{9}$.
\\Now we solve the systems of congruences.
\\
\\$x\equiv1\pmod{4}$,
\\$x\equiv4\pmod{9}$.
\\Using the first congruence,
\\Let $q \in \mathbb{Z}$ where $x=4q+1$.
\\Substituting into the second congruence,
\\$4q + 1 \equiv 4 \pmod{9}$.
\\It follows that $q \equiv (\bar{4})(3) \equiv 3 \pmod{9}$.
\\Let $r \in \mathbb{Z}$ where $q=9r+3$.
\\Now we have $x=4(9r+3)+1=36r+13$.
\\
\\$x\equiv1\pmod{4}$,
\\$x\equiv8\pmod{9}$.
\\Using the first congruence,
\\Let $q \in \mathbb{Z}$ where $x=4q+1$.
\\Substituting into the second congruence,
\\$4q + 1 \equiv 8 \pmod{9}$.
\\It follows that $q \equiv (\bar{4})(7) \equiv 4 \pmod{9}$.
\\Let $r \in \mathbb{Z}$ where $q=9r+4$.
\\Now we have $x=4(9r+4)+1=36r+17$.
\\
\\Therefore, the solution set is $\{x \mid x \equiv 13 \pmod{36}$
or $x \equiv 17 \pmod{36}\}$

\end{document}
