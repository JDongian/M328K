\documentclass{article}
\usepackage{amssymb}
\usepackage{amsmath}
\usepackage{centernot}

\begin{document}

\title{M328K\: Homework 6}
\author{Joshua Dong}
\date{\today}
\maketitle


2(e), 14(b,d), 16, 18 

% 4.2, problem 2(e)
\subsection{4.2.2.e}
Find all solutions to $128x \equiv 833 \pmod{1001}$
\\
\\$(128, 1001) = (128, 1001-8(128)) = (128, -23) = 1$
\\$\therefore$ there is exactly one unique solution mod 1001.
\\$128x - 1001q = 1
\\1001 + (-8)128 = -23
\\128 + (5)((1)1001 + (-8)128) = 13
\\(1)1001 + (-8)128 + (2)((1)128 + (5)((1)1001 + (-8)128)) = 3
\\(128 + (5)((1)1001 + (-8)128)) + (-4)((1)1001 + (-8)128 + (2)((1)128 + (5)((1)1001 + (-8)128))) = 1
\\128 + (5)1001 + (-40)128 + (-4)1001 + (32)128 + (-8)128 + (-40)1001 + (320)128 = 1
\\(-39)1001 + (305)128 = 1
\\\therefore \bar{128}=305$.
\\$\bar{128}(128x) \equiv \bar{128}(833) \pmod{1001}$.
\\$\therefore x \equiv 305(833) \equiv 812 \pmod{1001}$
\\$\therefore x \in S$ where $S = \{n \mid n = 812 + 1001k$\;\;$ \forall k \in \mathbb{Z}\}$.

\subsection{4.2.14.b}
$2x + 4y \equiv 6 \pmod{8}$
\\
\\$x + 2y \equiv 3 \pmod{4}$
\\$(1, 2, 4) \mid 3$
\\$4n = (x + 2y - 3)$ for some $n \in \mathbb{Z}$.
\\$(x + 2y - 4n) = 3$.
\\Let k be some integer where $2k = (2y - 4n)$.
\\Then $(x + 2k) = 3$.
\\By observation, we see that one solution is $x=3,\,k=0$.
\\All solutions to the previous equation then are expressible in the form:
\\$x=3-2t, k=t$\;\;$\forall t \in \mathbb{Z}$.
\\Now we solve $2k = (2y - 4n)$.
\\$k = (y - 2n)$.
\\First we solve $y-2n = 1$
\\By observation, we see that one solution is:
\\$y=3, n=1$
\\So the solution to $k = (y - 2n)$ is:
\\$y=3k+2s, n=k-s$\;\;$\forall s \in \mathbb{Z}$.
\\$\therefore x=3-2t, y=3t+2s$\;\;$\forall s,t \in \mathbb{Z}$ represent all the solutions to $x + 2y \equiv 3 \pmod{4}$.

\subsection{4.2.14.d}
$10x + 5y \equiv 9 \pmod{15}$
\\
\\$(10, 5, 15) \nmid 9$
\\Thus, there are no solutions.

\subsection{4.2.16}
Show $x^2 \equiv 1 \pmod{2^k}$ has exactly four unique solutions: $\pm1, \pm(1+2^{k-1})$ when $k>2$. Also show the cases for $k \equiv 1, k\equiv 2$.
\\
\\\emph{If $k=1$:}
\\$x^2 \equiv 1 \pmod{2}$
\\$\therefore 2 \mid (x^2 - 1)$
\\$\therefore 2 \mid (x + 1)(x - 1)$.
\\If x is even, then the product $(x + 1)(x - 1)$ is odd and not divisible by 2.
\\If x is odd, then the product $(x + 1)(x - 1)$ is even and divisible by 2.
\\$\therefore$ x must be odd.
\\The set of odd numbers congruent mod 2 less than 2 is \{1\}.
\\$\therefore x \in S$ where $S = \{n \mid n \equiv 1\}$.
\\$\therefore$ There is one unique solution if $k=1$.
\\
\\\emph{If $k=2$:}
\\$x^2 \equiv 1 \pmod{4}$
\\$\therefore 4 \mid (x^2 - 1)$
\\$\therefore 4 \mid (x + 1)(x - 1)$.
\\If x is even, then the product $(x + 1)(x - 1)$ is odd and not divisible by 4.
\\If x is odd, then $(x + 1)$ is even and $(x - 1)$ is even.
\\The product of two numbers divisible by 2 is divisible by 4.
\\$\therefore$ If x is odd, then 4 divides the product $(x + 1)(x - 1)$.
\\$\therefore$ x must be odd.
\\The set of odd numbers congruent mod 4 less than 4 is \{1, 3\}.
\\$\therefore x \in S$ where $S = \{n \mid n \equiv 1$ or $n \equiv 3\}$.
\\$\therefore$ There are two unique solutions if $k=2$.
\\
\\\emph{If $k>2$:}
\\$x^2 \equiv 1 \pmod{2^k}$
\\$\therefore 2^k \mid (x^2 - 1)$
\\$\therefore 2^k \mid (x + 1)(x - 1)$.
\\If x is even, then the product $(x + 1)(x - 1)$ is odd and not divisible by any power of 2 greater than 0.
\\If x is odd, then the product $(x + 1)(x - 1)$ is even and divisible by 2.
\\$\therefore$ any solutions that exist must be even.
\\Let n be an integer where $2n+1 = x$.
\\Then $2^k \mid ((2n+1) + 1)((2n+1) - 1)$.
\\$\therefore 2^k \mid (2n+2)(2n)$
\\$\therefore 2^k \mid (4)(n)(n+1)$.
\\Since $k>2$,
\\$2^{k-2} \mid (n)(n+1)$.
\\$m2^{k-2} = (n)(n+1)$ for some integer m.
\\\emph{If $m=0$, then:}
\\$n=0$ or $n+1=0$.
\\$\therefore n \in \{-1, 0\}$ provides all solutions where $m=0$.
\\$\therefore x \in \{-1, 1\}$ provides all solutions where $m=0$.
\\\emph{If $m \neq 0$, then:}
\\If n is even, then n+1 is odd. 
\\If n is odd, then n+1 is even. 
\\$2^{k-2} \mid (n)(n+1)$.
\\$\therefore$ either $2^{k-2} \mid n$ or $2^{k-2} \mid (n+1)$.
\\$\therefore x \in S$ where $S = \{n$ such that $2^{k-2} \mid n$ or $2^{k-2} \mid (n+1)$\;\;$\forall k \in \mathbb{Z}, k>2\}$ provides all solutions where $m \neq 0$.
\\$\therefore n \in S$ where $S = \{n$ such that $n = t2^{k-2}$ or $n = t2^{k-2} - 1$\;\;$\forall k,t \in \mathbb{Z}, k>2\}$ provides all solutions where $m \neq 0$.
\\$\therefore x \in S$ where $S = \{x$ such that $x = t2^{k-1}+1$ or $x = t2^{k-1}-1$\;\;$\forall t,k \in \mathbb{Z}, k>2\}$ provides all solutions where $m \neq 0$.
\\$\therefore x \in S$ where $S = \{x$ such that $x \equiv 2^{k-1}+1 \pmod{2^k}$ or $x \equiv 2^{k-1}-1 \pmod{2^k}$\;\;$\forall k \in \mathbb{Z}, k>2\}$ provides all solutions where $m \neq 0$.
\\$\therefore$ In the general case where $m \in \mathbb{Z}$, $x \in S$ where $S = \{x$ such that $x=-1$ or $x=1$ or $x \equiv 2^{k-1}+1 \pmod{2^k}$ or $x \equiv 2^{k-1}-1 \pmod{2^k}$\;\;$\forall k \in \mathbb{Z}, k>2\}$ provides all solutions where $m \neq 0$.

\subsection{4.2.18}
a
\\


\end{document}
