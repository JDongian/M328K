\documentclass{article}
\usepackage{amssymb}
\usepackage{amsmath}
\usepackage{centernot}
\setlength{\parindent}{0in}

\begin{document}

\title{M328K\: Homework 12}
\author{Joshua Dong}
\date{\today}
\maketitle

\subsection{9.1.12}
Show that if $n \in \mathbb{Z}^+$, for some $a, b \perp n$ such that
$ord_n a \perp ord_n b$, then $ord_n (ab) = ord_n a \cdot ord_n b$.
\\

\subsection{9.1.16}
Show that if $a, m \in \mathbb{Z}^+$, $a \perp m$ such that
$ord_m a = m-1$, then $m \in \textbf{P}$.
\\
\\In the previous problem, 9.1.12, we showed that the order function modulo some integer power is multiplicative across coprime factors.
\\Let $ord_m a = g(a)$ for convinience.
\\$g(a) = g(p_1^{a_1}p_2^{a_2}...p_{\omega(n)}^{a_\omega(n)}) = g(p_1^{a_1})g(p_2^{a_2})...g(p_{\omega(n)}^{a_\omega(n)})$ (by the fundamental theorem of arithmetic and the multiplicative property of the order function across coprime factors)
\\The order of prime powers is known. $g(p_i^{a_i}) = (a_i)g(p_i)$, if $\exists g(p_1)$.
\\If $a \notin \textbf{P}$, then the product derived from the fundamental theorem of arithmetic is not equal to a-1.
\\Therefore, if $a, m \in \mathbb{Z}^+$, $a \perp m$ such that $ord_m a = m-1$, then $m \in \textbf{P}$.
\\We can also observe that if m is prime, then the product derived from the fundamental theorem of arithmetic is always equal to $g(p_i)$, which is always $\varphi(m) = m-1$ when m is prime, by Fermat's little theorem.

\subsection{9.2.8}
Let $r$ be a primitive root of the prime $p$ with $p \equiv 1 \pmod4$.
Show that $-r$ is also a primitive root.
\\
\\We observe that r must be odd, because if $2 \perp r$,
then $r \not\perp 4$ and thus r cannot be a primitive root. Also, $p > 2$, so we do not have to worry about edge cases.
\\If we can prove that the order of $-r$ modulo p exists, then that is sufficent to prove that $-r$ is a primitive root.
\\By 9.1.12, we have that if $r \perp ord_n -1$, then $ord_n(-r) = ord_n(r) \cdot ord_n(-1)$.
\\$ord_n(-1) = 2$, because $-1\cdot-1=1$. We do not have to worry about moduli less than or equal to 2.
\\r is a primitive root, $\therefore \exists ord_n(r)$.
\\$\therefore \exists k \in \mathbb{Z}$ such that $k = ord_n(-r) = ord_n(r) \cdot ord_n(-1)$.
\\Therefore $-r$ is a primitive root.

\subsection{9.2.12}
Find the least positive residue of the product of a set of $\varphi(p-1)$
incongruent primitive roots modulo some prime $p$.
\\

%how to show all numbers whos totient is 12, exactly.

\end{document}
