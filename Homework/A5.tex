\documentclass{article}
\usepackage{amssymb}
\usepackage{amsmath}
\usepackage{centernot}

\begin{document}

\title{M328K\: Homework 5}
\author{Joshua Dong}
\date{\today}
\maketitle

% 3.7, problem 8
\subsection{3.7.8}
$18x + 33y = 549$ where $x,y > 0$ and $x,y \in \mathbb{Z}$
\\Find: $\min(x+y)$
\\
\\We first find $x, y > 0$ such that $18x + 33y = (18, 33) = 3$
\\By inspection, $18(2)+33(-1)=3$ is a solution to the above equation.
\\$549=183(3)
\\=183(18(2)+33(-1))
\\=(183)18(2) + (183)33(-1)
\\=18(366)+33(-183)$.
\\$\therefore \forall n \in \mathbb{Z}, 549=18(366-33(n))+33(-183+18(n))$.
\\$\therefore\ 549=18(3)+33(15)$ when $n=11$
\\$n>11 \rightarrow x<0$ and $n<11 \rightarrow y<0$.
\\$\therefore\ n=11, x=3, y=15$.
\\$\therefore \min(x+y)=3+15=18$.

% 4.1, problem 26
\subsection{4.1.26}
Show that if $a^k \equiv b^k \pmod{m}$
and $a^{k+1} \equiv b^{k+1} \pmod{m}$
where $a, b, k, m \in \mathbb{Z}$ with $k,m>0$
such that $(a,m) = 1$,
then $a \equiv b \pmod{m}$.
Is $(a,m) = 1$ required to show this?
\\
\\$a^{k+1} \equiv\ b^{k+1} \pmod{m}
\\a \cdot\ a^k \equiv\ b \cdot\ b^k \pmod{m}
\\a \cdot\ a^k \equiv\ b \cdot\ a^k \pmod{m}$.
\\$a^k \perp\ m \leftrightarrow a \nmid m$ by the fundamental theorem of arithmetic.
\\$(a,m)=1$.
\\$\therefore a \nmid m$
\\$\therefore\ a^k \perp\ m$
\\$a^k \perp\ m \rightarrow\ a \equiv\ b \pmod{m}$ by modular division by numbers coprime to the modulus.
\\If $(a,m)\neq1$ then modular division would be prohibited and the result could not be shown.


% Problem 3
\subsection{4.1.30}
something
% Problem 4
\subsection{4.1.34}

\end{document}
