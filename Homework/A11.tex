\documentclass{article}
\usepackage{amssymb}
\usepackage{amsmath}
\usepackage{centernot}
\setlength{\parindent}{0in}

\usepackage{color}
\usepackage{listings}
\lstset{ %
    language=Python,                % choose the language of the code
    basicstyle=\footnotesize,       % the size of the fonts that are used for the code
    numbers=left,                   % where to put the line-numbers
    numberstyle=\footnotesize,      % the size of the fonts that are used for the line-numbers
    stepnumber=1,                   % the step between two line-numbers. If it is 1 each line will be numbered
    numbersep=5pt,                  % how far the line-numbers are from the code
    backgroundcolor=\color{white},  % choose the background color. You must add \usepackage{color}
    showspaces=false,               % show spaces adding particular underscores
    showstringspaces=false,         % underline spaces within strings
    showtabs=false,                 % show tabs within strings adding particular underscores
    frame=single,           % adds a frame around the code
    tabsize=2,          % sets default tabsize to 2 spaces
    captionpos=b,           % sets the caption-position to bottom
    breaklines=true,        % sets automatic line breaking
    breakatwhitespace=false,    % sets if automatic breaks should only happen at whitespace
    escapeinside={\%*}{*)}          % if you want to add a comment within your code
}

\begin{document}

\title{M328K\: Homework 11}
\author{Joshua Dong}
\date{\today}
\maketitle

\subsection{8.1.8}
We count the word frequencies using Python. 'V' is the most common character. Listing all the rotations of the text, we also see that the rotation of +9 yeilds "THEVA LUEOF THEKE YISSE VENTE EN".
\\
\begin{lstlisting}
cipher = "KYVMR", "CLVFW", "KYVBV", "PZJJV", "MVEKV", "VE"
freq = {}
for c in ''.join(cipher):
    if c in freq.keys():
        freq[c] += 1
    else:
        freq[c] = 1 
print sorted(freq.items(), key=lambda x: x[1])
def rot(w, i): 
    return ''.join(chr(((ord(c)-ord('A')+i)%26) + ord('A')) for c in w)
for i in xrange(26):
    print ' '.join(rot(w, i) for w in cipher)
\end{lstlisting}

\subsection{8.1.12}
Using the same program as before, we produce this output:
\\('A', 1), ('E', 1), ('K', 1), ('J', 1), ('O', 1), ('V', 1), ('D', 2),
\\('G', 2), ('I', 2), ('N', 2), ('S', 2), ('U', 2), ('Z', 2), ('C', 3),
\\('X', 3), ('W', 4), ('P', 5), ('Y', 5), ('R', 6), ('M', 7)
\\We can now guess that 'M' corresponds to 'E' and 'R' corresponds to 'T'.
\\$c(ord(M)-d) \equiv ord(E) \pmod{26}$
\\$c(12-d) \equiv 4 \pmod{26}$
\\$12c-cd \equiv 4 \pmod{26}$
\\
\\$c(ord(R)-d) \equiv ord(T) \pmod{26}$
\\$c(17-d) \equiv 19 \pmod{26}$
\\$17c-cd \equiv 19 \pmod{26}$
\\
\\$5c \equiv 15 \pmod{26}$
\\$(-5)5c \equiv (-5)15 \pmod{26}$
\\$-25c \equiv -75 \pmod{26}$
\\$c \equiv 3 \pmod{26}$
\\$3(12-d) \equiv 4 \pmod{26}$
\\$36-3d \equiv 4 \pmod{26}$
\\$-3d \equiv -6 \pmod{26}$
\\$d \equiv 2 \pmod{26}$
\\
\\Now we use a similar program to decode the text:
\begin{lstlisting}
def decode(C, c, d):
    return ''.join(chr( ( c*((ord(i)-ord('A')-d) )%26) + ord('A')) for i in C)
print decode(cipher, 3, 2)
\end{lstlisting}
This produces:
\\EVERYALCHEMISTOFANCIENTTIMESKNEWHOWTOTURNLEADINTOGOLD

\subsection{8.1.8}
We keep guessing the factorization for 2881 until we find that $2881=(43)(67)$.
\\
\\We can find $\varphi(2881) = \varphi(43)\varphi(67) = 42*66 = 2772$
\\Now we find the inverse of 5 modulo 2772.
\begin{lstlisting}
for i in xrange(2772):
    if (5*i)%2772 == 1:
        print i
\end{lstlisting}
This returns 1109, the value of d.
\\It is now trivial to decode the cipher:
\begin{lstlisting}
import re
ciphertext = '05041874034705152088235607360468'
def decode(C, d, n): 
    output = ""
    for c_i in re.findall('..?.?.?', C): 
        c_i = str((int(c_i)**d)%n).zfill(4)
        output += chr(int(c_i[:2])+ord('a'))
        output += chr(int(c_i[2:])+ord('a'))
    return output
print decode(ciphertext, 1109, 2881)
\end{lstlisting}
This produces the string 'eatchocolatecake'.

\newpage
\subsection{8.1.14}
If $n_1, n_2, n_3$ are not coprime, then factoring them is trivial and thus the key is easily produced.
\\If $n_1, n_2, n_3$ are coprime, then we have a system of congruences:
\\$x_1 \equiv P^3 \pmod{n_1}$
\\$x_2 \equiv P^3 \pmod{n_2}$
\\$x_3 \equiv P^3 \pmod{n_3}$
\\By the Chinese Remainder Theorem, there must exist some $y \in \mathbb{Z}$
where $y \equiv P^3 \pmod{n_1n_2n_3}$
\\Since RSA requires that $P < n_i$, we know that $P^3 < n_1n_2n_3$.
\\$\therefore y = P^3$.
\\Thus deciphering is reduced to solving the set of linear congruences and then finding the cube root of the solution. This is significantly easier than factoring large numbers.
\\(Note: we can make the same argument for any $e$, meaning that having $e$ ciphertexts with their corresponding $n_i$ will lead to an easier computation to determine the plaintext)

\end{document}
